\documentclass{memoir}
%\usepackage[margin=2in]{geometry}
\usepackage[osf,p]{libertinus}
\usepackage{microtype}
\usepackage[pdfusetitle,hidelinks]{hyperref}

\usepackage[series={A,B,C}]{reledmac}
\usepackage{reledpar}

\usepackage{graphicx}
\usepackage{polyglossia}
\setmainlanguage{english}
\setotherlanguage{hebrew}
\gappto\captionshebrew{\renewcommand\chaptername{קאַפּיטל}}
\usepackage{metalogo}


%%linenumincrement*{1}
%%\firstlinenum*{15}
%%\setlength{\Lcolwidth}{0.44\textwidth}
%%\setlength{\Rcolwidth}{0.44\textwidth}

\begin{document}
%%\maxhnotesA{0.8\textheight}
\renewcommand{\abstractname}{\vspace{-\baselineskip}}
\title{Extracts from Labzik}
\author{by Hava Pava \\ \\ Transl. Ilan Pillemer}
\date{\today}

\maketitle

\tableofcontents

\begin{pairs}

\begin{Rightside}

\begin{RTL}
\begin{hebrew}
\firstlinenumR{10000000}
\beginnumbering
%%\numberpstarttrue

\autopar

נעמט אַ פֿרוי פֿון בראָזװיל איר הינטעלע און לײגט דאָס אַרײן
אין אַ קויש, קעדײ ס'זאָל ניט קענען זען דעם װעג. אָבער לאַבזיק
פֿרײט זיך, װאָס מען לײגט דאָס אין אַ קויש.  ליגט דאָס דאָרטן מיטן
בײכעלע אַרויף און װאַרפֿט מיט די פֿיסעלעך און קעװלט אָן.

גײט אַרויף די פֿרוי אויף סאָטער עװעניו סטײשאָן, װאַרפֿט אַרײן
אַ ניקל און לויפֿט אַרויף אויבן. װאַרט שוין אַ באַן, װאָס לויפֿט מיט
זײ גיך־גיך איבער אַלע דעכער פֿון די הײזער און דערנאַכדעם אונטער
דער ערד, און נאָכאַמאָל אויבן איבער אַלע דעכער, און װײטער אַמאָל
אונטער דער ערד.

װי אַזוי די באַן פֿאָרט רעדט דערװײל די פֿרוי צום הינטעלע,
װאָס אין קויש.

„און לאַבזיק זײ ניט בײז אַף מיר, װאָס איך טו אזוי. ס'איז צוליב דעם קריזיס.“

„האַװ, האַװ,“ מאַכט לאַבזיק און װיל איר אַ לעק טאָן מיטן
צינגל אָן איר שפּיץ נאָז.

לאָזט זיך ניט די פֿרוי אַ לעק טאָן אָן דער נאָז רעדט װײטער:
„און מײן ניט, לאַבזיק, אַז איך האָב דיך פֿײנט. איך האָב  דיך
ליב, נאָר סיאיז ניטאָ דיר װאָס צו געבן צום עסן דערפֿאַר טאַקע
האָבן מיר אויך אַװעקגעשיקט אונזער קלײנע טאָכטער עמאַן צוּ דער 
טאַנטע קײן באָסטאָן.

„האַװ, האַװ“ ענטפֿערט לאַבזיק און װיל אַ כאַפּ טאָן איר
פֿינגער מיט די צײנדעלעך.

לויפֿט די באַן איבער אַלע דעכער, לויפֿט די באַן אונטער דער
ערד און אַזוי קומען זײ אָן אין די בראָנקס צו דזשעקסאָן עװענוי
סטײשאָן. שטעלט זיך דאָרטן אָפּ די באַן אַף אײן מינוט. הויבט זיך
אויף די פֿרוי פֿון בראָנזװיל. גיט זי אַ כאַפּ לאַבזיק פֿון קויש און
שטעלט אים אַרוים אַף דער פּלאַטפֿאָרמע און אַלײן בלײבט זי, די
פֿרוי פֿון בראָנזװיל, אינעװײניק אין דער באַן. די טירן פֿאַרמאַכן זיך
און דאָס הינטעלע לויפֿט נאָך און שרײט מיט אַלע קויכעס:

„האַװ, האַװ, לאָז מיר ניט איבער, לאָז מיר ניט איבער.“

אָבער ס'איז שוין פֿאַרפֿאַלן. די באַן איז אַנטלאָפֿן מיט דער פֿרוי
פֿון בראָנזװיל און דאָס הינטעלע איז איבערגעבליבן אַלײן אין אַ
פֿרעמדער שטאָט מיטן צעטעלע אַפֿן האַלדז.

איז פֿריער געװען אַזוי: לאַבזיק האָט געמײנט, אַז ס'איז אין
שפּאַס בלויז, און באַלד װעט מען דאָס קומען אָפּנעמען. האָט זיך דאָס אַנידערגעלײגט אין אַ
 װינקעלע און געװאַרט.

װאַרט דאָס און װאַרט און דער טאַג פֿאַרגײט אַזוי און ס'קומט
אָן די נאַכט. הײבט דאָס אָן צו װײנען אין דער שטיל.

„װאו, װאו, װאו.“

לויפֿט אַלעמאָל אָן אַ באַן, עפֿענען זיך די טירן,
לויפֿן זײ אינגיכן אַרוים און שפּרינגען אַראָפּ דער טרעפּ. אַף לאַבזיך
קוקט זיך אָבער קײנער ניט אום.

הויבט דאָס שוין אָן צו װײנען אינדערהויך:

„װאַ־ו,  װא־ו.“

לויפֿט װידער אָן אַ באַן, צעעפֿענען זיך די טירן מיטאַמאָל, לויפֿן
אַרוים אַרבעטער און שפּרינגען אויך אַראָפּ די  טרעפּ. 

הויבט זיך שוין לאַבזיק אויף. שפּאַנט זיך איבער דער פּלאַטפֿאָרמע
אַהין און צוריק מיט אַראָפּגעלאָזטע אויערן אוּן װײנט:

װ ־ א ־ ו.

דערװײל הײבט אָן צו פֿאַלן אַ רעגן. ס'װערט קאַלט. באַנען
צו לוּיפֿן, מענטשן גײען אַרויס, אָבער קײנער קוקט זיך ניט אום אַפֿן
הינטעלע מיטן צעטעלע אַפֿן האַלדז.

טראַכט שוין לאַבזיק אַז ס'װעט נאָך שטאַרבן און אַז ס'װעט זײן
זײער שלכט, אַז ס'װעט שטאַרבן, האָט דאָס שטאַרק ראַכמאָנעס אַף
זיך אַלײן און כניקעט װי אַ קלײן קינד:

„אי, אי, אי.“

װעלן מיר דאָס איבערלאָזן אַף דער פּלאַטפֿאָרמע װי ס'כניקעט
אי, אי, אי, און מיר װעלן דערװײל דערצײלן פֿון בערל דעם אַפּרײטער. 

בערל דער אַפּערײטער האָט גרויסע שװאַרצע אויגן, װאָס לאַכן
שטענדיק. ער האָט אויך אַ װײב אַ בעריע, װאָס הײסט מאַלי, אַ קלײן
אינגל, װאָס הײסט מוליק און אַ קלײן מײדעלע, װאָס הײסט ריפֿקעלע.
איז ער זיך געפֿאָרן אין באַן, געזעסן לעבן פֿענצטער און ארויסגעקוקט.

„אַך“, מאַכט ער, „ס'גײט אַ רעגן, אָבער אַ קאַפּאָרע דער רעגן.
אַבי ס'װעט זײן אַ גוטע װעטשערע.

שטעלט זיך אָפּ דער לעקסינגטאָן עװעניו עקספּרעס לעבן דזשעקסאָנ
עװעניו, לויפֿט אַרוים בערל דער אַפּרײטער מיט די לאַכנדיקע
אויגן און װיל זיך שוין לאָזן אַהײם, דערזעט ער אָבער באַלד דאָס
הינטעלע מיטן צעטעלע אַפֿן האַלדז, װאָס ציטערט פֿון קעלט און
כניקט װי אַ קינד:

„אי, אי, אי.“

בויגט ער זיך אָן צו דעמ לעקט דאָס בערעלן די פֿינגער. געפֿעלט
עס בערעלן, װאָס מען לעקט אים די פֿינגער. לײנט ער איבער דאָס
צעטעלע און ס'שטײט דאָרטן אָנגעשריבן אַזוי: ס'איז אַ קריזיס. מײן
מאַן אַרבעט ניט. און מיר האָבן עס ניט װאָס צו געבן עסן. גוטע
מענטשן, װאָס האָבן נאָך אַרבעט, נעמט דאָס אַרײן. ס'הײסט „לאַבזיק“.“

גיט בערל אַ טראַכט צו זיך:

„זאַל איך דאָס נעמען אַהײם, אָדער זאָל איך דאָס ניט נעמען.
מען װעט דאָך דאָס דאַרפֿן געבן עסן און איך אַרבעט אַזוי װינציק.
אָבער אַ קאַפּאָרע די װינציק אַרבעט. איך װעל דאָס נעמען אַהײם.

װיקלט ער דאָס אײן אין דער „מאָרגן־פֿרײהײט“ און לאַבזיק לאָזט
זיך. סע װײסט שוין, אַז עס האָט שוין אָנגעטראָפֿן אַ גוטן מענטשן.

דאָ דאַרף איך שוין ניט דערצײלן די גרויסע פֿרײד פֿון ריפֿקעלען
און מוליקן װען זײער טאַטע בערל דער אַפּרײטער האָט אַרײנגעבראַכט
אין הויז דאָס קריזיס־הינטעלע. זײ האָבן דאָס געגעבן װאַראַמער מילך.
זײ האָבן דאָס אויסגעװאַשן. זײ האָבן דאָס אויסגעטריקנט מיט אַ
האַנטעך און דערנאָכדעם דאָס געלײנט שלאָפֿן. ריפֿקעלע האָט געװיגט
דאָס װינגעלע און אונטערגעזונגען:

„אָ, לאַבזיק, אָ, שלאָף הינטעלע פֿון קריזיס, שלאָף. די פּיאָנערן
װעלן שוין אַװעקטרײבן דעם קריזיס. אָ, שלאָף, שלאָף.

האָט לאַבזיק טאַקע געפֿאַלגט און איז אײנגעשלאָפֿן. 

\endnumbering
\end{hebrew}
\end{RTL}
\end{Rightside}


\begin{Leftside}
\begin{english}
\chapter{
The crisis dog. \\  \RLE{
דאָס קריזיס הינטעלע.
}  }
\newpage
\beginnumbering
\autopar

A woman from Brozville \footnoteA{Bronxville, perhaps?} takes her little dog and puts him inside a basket,
so that he is not able to see the journey. Nevertheless, Labzik takes joy that someone is putting him
in a basket. Thus, he is lying there with his little tummy upwards and waving his little feet and full of delight.

The woman goes up to Soter Avenue Station, throws in a nickel, and runs up above. A train is already waiting,
which speeds along with them click-a-clack over all the rooftops of the houses and then under the ground,
and then once again up and above all the rooftops, and later again under the ground.

And meanwhile, as the train travels, the woman speaks to the little dog that is in the basket. 

``And Labzik, don't be cross with me, that I am doing this. It is on account of the crisis.''

``Woof, woof," goes Labzik and wants to lick with his little tongue the tip of her nose.

The woman, not allowing her nose to be licked, continued speaking :
``And do not think, Labzik, that I dislike you. I love you, but there is nothing to give you to eat, and
because of this we even have also sent our small daughter, Emen, away to her Aunt in Boston.''    

``Woof, woof'' replies Labzik and wants to nibble her finger with his little teeth.

The train speeds over all the rooftops, the train speeds under the ground and so 
they arrive in the Bronx, at the Jackson Avenue Station. The train waits there
for one minute. The lady from Bronxville gets up. This is in order to quickly take
Labzik from the basket and put him out on the platform; and she herself remains, the woman
from Bronzville, inside the train. The doors are closing and the little dog is running
still and calling with all his might:

``Woof, woof, don't leave me behind, don't leave me over here.'' 

But it has already happened. The train has left with the woman from 
Bronz-ville and the little dog has been left behind, alone, in a strange city 
with a note attached to his neck.

At first it was so: Labzik thought that it was only a joke, and soon someone will
come to fetch him. Therefore he laid himself down in a corner and waited.

He waits and waits and the day passes so and the night arrives. And he begins to whimper
in the silence.

``Ooo, Ooo, Ooo."

There is always a train with the working class rushing in. Then opening its doors,
and soon they are all around and jumping down the steps. But no-one notices Labzik.

He already is beginning to sob in the catch of his breath. 

``OO-oo, OO-oo"

Again a train is rushing in, the doors suddenly opening, workers hurrying around,
and also jumping down the steps.

Labzik is already getting up. He walks along the platform, there and back, with 
drooping ears and whimpering.

"OO-OO-OO"

Meanwhile rain begins to fall. It is becoming cold. Trains are coming
in a rush, people are moving all around, but no-one notices the little
dog with a note attached to his neck.

Then Labzik is considering that he will just pass away, and that this will be
really terrible if he will die, and he felt deep pity for his lonesome self and
wimpers like a small child: 

``Iii, iii, ,iii"

Now, we will leave behind the platform on which there
is an ``iii, iii, iii'' and we will meanwhile tell of Beryl the operator.

Beryl the operator has large black eyes which are always laughing. He 
also has an extremely efficient wife who is called Mali; a small boy who is called 
Mulik and a small girl, who is called Rifkele. He had travelled in the train, he had
sat near the window and looked outside

``Ach", he is muttering, "It is raining, oh but what a worthless rain. At least there will
be a good dinner.''

The Lexington Avenue Express pulls up by Jackson Avenue, and then Beryl the Operator with the 
laughing eyes rushes, wanting to already dash home, but nevertheless he quickly notices the
little dog with the little note attached to his neck who is shivering from cold and whimpering like 
a child. 

``Iii, iii, ,iii"

He is stooping down towards him, and then he licks Beryl's fingers. Beryl enjoyed it when
his fingers are being licked. Then he read over the note and on it there was written as so:
``It is a crisis. My husband has no work. And we dont have anything to give to eat. Good people,
who still have work, take him in. He is called ``Labzik''.''

Beryl quickly thinks through it:

``Should I take him home, or should I not take him. One will surely need to feed him
and I have such meagre work. But so what that the work is meagre. I will take him home.''

He then swaddles him in the "Morning-Freedom" and Lazik allows this. Its already clear that he has
just happened upon a good person.

I indeed do not need to tell you the great joy of Rifkele and Mulik when their father Beryl the operator
brought the crisis-dog into the house. They gave him warm milk. They washed him. They dried him with
a towel, and after this they lay him down to sleep. Rifkele rocked him a little and whispered:

``Oh, Labzik, Oh, sleep little dog from the crisis, sleep. The new beginnings will soon drive out the crisis.
Oh, sleep, sleep.''

And then Labzik indeed took the advice and fell asleep.

\endnumbering
\end{english}
\end{Leftside}

\end{pairs}
\Columns


\end{document}



















































